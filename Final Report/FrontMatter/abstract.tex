This research investigates the potential of implementing a Naive Bayes Classifier into a minimax based chess engine. Modern chess engines often rely on complex machine learning algorithms, this research explores whether a simpler algorithm can improve efficiency and achieve similar performance. This paper evaluated two different implementations, one where the minimax evaluation function is fully replaced by the Naive Bayes model and one that supports traditional evaluation functions. 

Over 2880 chess games were played, and the results of the experiments showed that the integration approach outperformed the substitution approach. It consistently achieved a higher win rate, superior piece management and faster move times. This study also investigated the influence of feature selection, finding that generally an increase number of features improves performance, however, overly complicated features could lead to weakened performance. The use of random sample games for training proved to generalise better than those trained on only master level games or only beginner level games, highlighting the importance of a diverse training set. 

Overall, the findings indicate a potential for Naive Bayes to be used alongside other evaluation techniques in chess engines. Future research could explore ensemble methods or the integration of other machine learning techniques to improve overall performance.