\chapter{Legal, Social, Ethical and Professional Issues}
\section{Legal Issues}

As mentioned before, for this project the python-chess library was primarily used for the implementation of the chess engine. This library is licensed under the GNU General Public License v3.0 (GPLv3) \cite{PythonchessPurePython}. This license is a free software license that allows developers to freely use, study and modify the python-chess library for their projects \cite{GNUGeneralPublic}. The license also requires that any modifications made to the library must be distributed under GPLv3, meaning that the source code must be available to the public which is fulfilled as the source code is publicly available on GitHub \cite{fiekasNiklasfPythonchess2025} \url{https://github.com/niklasf/python-chess}. 

Other libraries used in this project include numpy, pandas and joblib whcih are all licenced under the BSD 3-Clause License as well as scikit-learn which is licensed under the BSD License.

This research utilises the `Chess' dataset available on Kaggle \cite{ChessGameDataset} uploaded by Mitchell J. The dataset is open to be used by the public under the Creative Commons CCO 1.0 Universal license, meaning that this dataset can be freely copied, modified, distributed and used for any purpose without requiring permission from the creator. Despite not being legally required, we would like to acknowledge the contribution the creator has made to this research project and others. This data does not contain any directly identifiable information from Lichess users. Player usernames were included in the dataset but these do not directly reveal real-world identities. The data does not include sensitive personal data like real names, email addresses or phone numbers.

\section{Social Issues}

The chess engine developed in this project is a tool that can be used to help players improve their chess skills, however, it has been engineered for someone who has some technical ability. Understanding Python and basic command-line usage is required to run the engine. Also, the output of the engine is standard algebraic notation which most chess players are familiar with, but players can not gain an understanding of why the engine made a particular move. A GUI was implemented to help users visualise the board and the moves made by the engine. This GUI would be more beneficial paired with a more readable explanation of the engine's moves. In the future, the engine could be more accessible to a wider audience by implementing features like audio outputs for visually impaired users or support for other languages. Another feature that could be beneficial is an Open API that would allow developers to integrate the engine into their own applications, potentially leading to more innovative ways to use the engine and more research opportunities. 

The advancements and increased accessibility of machine learning-based chess engines could have major implications on the chess community. More powerful chess engines being very available could cause a reduction in demand for human chess coaches. These engines could provide personalised training, analyse moves and provide feedback to players, much better than a human coach may be able to do. This could lead to a decrease in people playing chess, especially at the professional level. However this is very unlikely to replace human coaches but rather the increase in availability of chess engines could have a positive impact since it could allow those who may not have the resources to have a coach, lowering the barrier to entry for the game. It can be used as an educational tool for players, generating training exercises, analysing games and explaining concepts.

\section{Ethical Issues}

An ethical advantage of using Naive Bayes over other machine learning techniques is its transparency and interpretability. Unlike models that are considered `black boxes' like Neural Networks, Naive Byes allows users to understand the reasoning behind the model's predictions. Users are more likely to trust the model if they can understand the engine's thinking process. A Naive Bayes chess engine wouldn't necessarily harm a person's life, it is the responsibility of developers to consider the ethical implications it could have. One main risk is the potential misuse of the engine, primarily in online gaming or competitions. For this reason, we encourage users to use the engine to use this tool for learning and analysis and strongly discourage any form of cheating and encourage fair play.

\section{Professional Issues}

This project was in line with the principles as mentioned in the Code of Conduct \& Code of Good Practice issued by the British Computer Society. I, Mohammad Ibrahim Khan, applied my knowledge and skills to the best of my ability and worked within my areas of competence and sought external guidance from my supervisor, Jeffery Raphael, where necessary. All data, results and conclusions presented in this report are accurate and truthful to the best of my knowledge. The intellectual property rights of others have been respected throughout, properly citing all external datasets, libraries and references. As discussed in previous sections, measures were taken to ensure the privacy of individuals. Usernames were used as pseudonyms and no attempt was made to identify individuals.  

A number of measures were taken to ensure the transparency and ease of use of the chess engine. A detailed explanation of how the Naive Bayes classifier evaluates chess positions was provided in this report including the features used and the training process. The limitations of the engine such as potential bias and inability to understand complex situations have been clearly communicated and potential ethical issues related to the engine were also discussed. 